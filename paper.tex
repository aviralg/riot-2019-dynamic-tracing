\documentclass[acmsmall]{acmart}
\settopmatter{printfolios=false,printccs=false,printacmref=false}
\setcopyright{none}

\usepackage[utf8]{inputenc}
\usepackage{todonotes}
\usepackage{paralist}
\usepackage{listings}
\usepackage{xspace}

\lstset{ %
  basicstyle=\small\ttfamily
}

\newcommand{\code}[1]{\lstinline|#1|\xspace}
\newcommand{\Ie}{\emph{i.e.}\xspace}
\newcommand{\RDT}{RDT\xspace}
\newcommand{\RDyntrace}{R-dyntrace\xspace}
\newcommand{\dynalyzer}{dynalyzer\xspace}

\title{RDT: A Dynamic Tracing Framework for R}

\author{Aviral Goel}
\affiliation{
  \institution{Northeastern University}
  \country{USA}
}
\email{goel.aviral@gmail.com}

\author{Filip Křikava}
\affiliation{
  \institution{Czech Technical University}
  \country{Czechia}
}
\email{filip.krikava@fit.cvut.cz}

\author{Jan Vitek}
\affiliation{
  \institution{Northeastern University}
  \country{USA}
}
\affiliation{
  \institution{Czech Technical University}
  \country{Czechia}
}
\email{j.vitek@neu.edu}

\authorsaddresses{}

\begin{document}

\maketitle

\vspace{-1mm}

Static analysis of R code is hard. A combination of dynamic typing, laziness
with side-effects, introspection, metaprogramming, first-class environments and
\code{eval} enable programming idioms that make it nearly impossible to
statically derive reliable insights. Fortunately, most of the R packages
available in CRAN and Bioconductor contain runnable programs in the form of
examples, tests and vignettes\footnote{The packages contain approximately 4.4
million lines of runnable R code} which makes the language well-suited for
dynamic analysis.
%
However, currently there is no dynamic tracing functionality in R other than the
coarse-grained \code{trace} function which only allows tracing the entry and exit
points of R closures.

In this talk we present \RDT, a scalable and modular dynamic tracing framework
that exposes many aspects of R, facilitating a fine-grained inspection of
programs. It has two components:
%
\begin{compactitem}[$-$]
\item \emph{\RDyntrace} is a small extension\footnote{The complete extension to
    the virtual machine is about 1.8K lines of C code.} to GNU R Virtual Machine
  with probes that are triggered on specific program execution events. There are
  probes for function entry and exit, variable definition and mutation, garbage
  collection entry and exit, non-local jumps, promise creation and execution,
  \code{eval} entry and exit, and S3/S4 dispatch. \RDyntrace provides an API to
  register user-defined callbacks to these probes along with a facility to store
  arbitrary state per tracing session.
  %The ability to store tracing session state avoids the need to use
  %global variables and thus encourages a functional design of dynamic analysis
  %that is easier to reason about.
\item \emph{\dynalyzer} is a companion R package that provides utilities for
  building data-intensive dynamic analyses. The package facilitates quick
  construction of multistage pipeline for analysis of tracing data using the
  map-reduce programming model. It contains functions to automate logging and
  memoization of partially analyzed data from each stage. It also defines a
  data format for fast storage and retrieval of collected data with support for
  compression and streaming.
\end{compactitem}

%\bigskip
\indent A dynamic analysis is written as a standalone R package. Its separation
from the framework simplifies development and fosters reuse. It contains code
that defines tracer state and registers callbacks for relevant events. When a
callback is triggered, it receives the tracer state and information about the
event, \Ie, affected R objects and the relevant R interpreter state. \RDyntrace
API exports utilities to access fields of these objects as well as the global R
interpreter state such as the evaluation order of builtin function arguments and
variable lookup without triggering callbacks. Furthermore, \RDyntrace can detect
nesting in callbacks, a consequence of the tracing algorithm inadvertently
executing R code that potentially modifies the program state. This is a common
stumbling block in na\"ive tracing attempts and our design captures these cases,
reporting them as meaningful error message. There is also an escape hatch in the
form of disabling specific probes when the R code in question is known to be
benign.

We have used \RDT for a large-scale study of the design and use of laziness in
R. Its design has enabled us to scale the dynamic analysis to 14,875 R packages,
tracking over 177.8 billion promises and 3.5 trillion events at the rate of 1.5
million events per second. The result is a comprehensive, data-oriented
understanding of the role of laziness in real-world R code.

\RDT enables a data-driven approach to understanding R program behavior and
language usage patterns. We believe this approach is instrumental in providing
insights for improving the design of existing features and driving the design of
new features and better libraries for the dynamic languages of today.
\end{document}
