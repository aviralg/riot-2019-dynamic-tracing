\documentclass{article}
\usepackage{geometry}
\geometry{a4paper, margin=1in}

\title{A Dynamic Tracing Framework for R}

\author{
  Aviral Goel                                                          \\
  Northeastern University, USA                                         \\
  \texttt{goel.aviral@gmail.com}
  \and
  Filip Krikava                                                        \\
  Czech Technical University, Czechia                                  \\
  \texttt{krikava@gmail.com}
  \and
  Jan Vitek                                                            \\
  Northeastern University, USA and Czech Technical University, Czechia \\
  \texttt{j.vitek@northeastern.edu}}

\date{}

\begin{document}

\maketitle

R occupies a unique place in the design spectrum of programming languages. It’s
a dynamically typed, vectorized, lazy, side-effecting, object-oriented and
functional programming language. Dynamic dispatch, reflection (capturing the
call stack at runtime), first class environments, \texttt{eval}, metaprogramming
using \texttt{substitute} and interop with C/C++, Java and Fortran enable programming
idioms that make it nearly impossible to derive reliable insights by
statically analyzing R code.

While static analysis of program behavior is far from reach, the availability of
over 16K R packages from CRAN and Bioconductor with runnable programs in the
form of test cases, examples and vignettes motivates the application of dynamic
analysis approaches to solve the same problems. Altogether, the packages contain
XXX lines of executable R code and YYYY lines of executable C/C++ code. This
provides an interesting opportunity for the design of dynamic tracing framework
for large scale language studies.

In this talk we present a scalable, extensible and modular dynamic tracing
framework for R. While R has a facility, \texttt{trace}, that supports injection
of debugging code at entry and exit points of R functions, it is too coarse
grained to capture the whole range of dynamic features such as side effects,
runtime code generation and dynamic dispatch. Our tracing framework builds upon
this idea and exposes all these aspects of R, facilitating a fine-grained
inspection of R programs. It comprises of two components:
\begin{itemize}
\item \textbf{RDT} is a modified R Virtual Machine. It adds probes (C callbacks)
  that are executed when specific events happen on program execution. Currently,
  the supported events are function entry and exit, variable mutation and
  definition, garbage collection entry and exit, non-local jumps, promise
  creation and execution, eval entry and exit, S3 and S4 dispatch, etc. These
  probes are exposed as an API through a C \texttt{struct} which also contains
  an opaque pointer for storing arbitrary tracer state. This facilitates the
  tracer to not create global state, encouraging a functional design.
\item \textbf{Dynalyzer} is an R package that provides a large scale map-reduce
  style multistage data analysis pipeline with extensive logging and memoization
  of intermediate analysis results. It also exports a collection of utilities to aid
  common dynamic analysis tasks such as a data format for fast storing and
  reading compressed binary streaming data and efficient computation of function
  identity. Furthermore, it also exports functions for common graphing tasks.
\end{itemize}

The actual tracing algorithm itself is written as an R package that installs
with RDT. This tracer package registers callbacks for the relevant events
through the \texttt{struct} exposed by \texttt{RDT}. When callback is triggered,
it is passed the tracer state as an argument along with the information about
the event. This information generally comprises of affected R objects along with
the relevant interpreter state. \texttt{RDT} also exports utility functions to
access fields of these objects and global R interpreter state such as the
evaluation order of builtins, or a side-channel function to lookup environments
without triggering callbacks. Furthermore, \texttt{RDT} can detect recursion in
callbacks, a consequence of the tracing algorithm inadvertently executing R code
that potentially modifies the program state. This is a common stumbling block in
many a tracing attempts and our design captures these cases and reports a
meaningful error message. Furthermore, \texttt{RDT} also provides an escape
hatch in the form of disabling specific probes when the R code in question is
known to be benign. Separation of the tracing framework into separate
packages/components that access each other via well-defined API facilitates easy
development and reuse.

We have used framework for a large scale study of the design and use of
laziness in R. Our design enabled us to track over XXXX Billion promises and
XXXX Billion events providing very reliable data-oriented understanding of
the role of laziness in real world R code. We believe that this approach of
understanding program behavior, language usage patterns and performance enabled
by our framework can help improve the design of existing features, drive the
design of new language features and provide inspiration for the design of better
libraries.


\end{document}
